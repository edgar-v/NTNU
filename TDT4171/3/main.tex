\documentclass{article}

\usepackage[utf8]{inputenc}
\usepackage{enumerate}
\usepackage{amsmath}
\usepackage{graphicx}
\usepackage{apacite}
\usepackage{blindtext}

\usepackage[margin=1cm]{caption}

\setlength{\parindent}{0.0in}
\setlength{\parskip}{0.1in}


\title{TDT4171 Artificial Intelligence Methods\\ Exercise 3}
\author{Edgar Vedvik -- edgarmv@stud.ntnu.no}
\date{2017-03-03}


\begin{document}
\maketitle

\begin{abstract}
    \blindtext
    
\end{abstract}

\section*{Introduction}
    In this exercise we were tasked with creating a decision support system for
    a decision problem of our own choice. The exercice listed some examples,
    and one of the examples was exactly a problem I was facing this week:
    \emph{Should I go out on Friday or stay home doing this exercise?} Next week
    I had many exercises that were due. I also had plans to go to an event
    Saturday evening. Therefore I had good reason to stay home. However, as
    we all know, staying home Friday night doing exercises, while all your
    friends are out, isn't much fun.
    
    The exercise required that we had to measure the success of our choice. I
    measured this as the quality of life I would achieve when making this
    choice. The exercise also required that this decision problem had to
    contain at least 10 variables, and that half of them would have to be
    uncertain at the time the decision was made.

    The decision problem was modelled in GeNIe, which is a graphical user
    interface for solving just such a problem. The interface was relatively
    easy to work with once I learned how to create nodes and connect them
    together and adding probabilities.

\section*{Model}

    Once I had chosen what decision problem to model, and my utility function.
    I followed the steps recommended in \citeA[pp. 634]{artificial} to create
    the model. The first step was to create a causal model. This meant making a
    dependency graph, with lines between each dependency. I identified the
    following variables as directly affecting my quality of life:
    \begin{itemize}
        \item Will I finish the exercise in time?
        \item Did I have a good time Friday night?
        \item The amount of money I have.
        \item My physical state (hungover / tired / well rested)
    \end{itemize}

    All of these variables are influenced by many other variables. Including
    all of the variables is not in the scope of this report, but I have
    included those I found most important:

    \begin{itemize}
        \item Exercise deadline.
        \item How much of the exercise have i already done (progress).
        \item Are my friends going out on Friday?
        \item Did I work this week?
        \item Did I go out on Thursday?
        \item Do I have plans on Saturday?
        \item Will I make those plans?
    \end{itemize}

    \begin{figure}[ht] \centering
        \centering
        \includegraphics[width=\linewidth]{drawing.pdf}
        \caption{The model of the decision problem. The green rectangles are
        observable variables.  The yellow ellipses are unknown at the time
        the decision is made, and the blue hexagon is the utility.}
    \label{fig:model} \end{figure}

    After adding all those variables and identifying how they relate to each
    other I ended up with the result shown in Figure \ref{fig:model}.

    The second step was to simplify and remove variables that did not affect the
    decision. Because my model was so simple and all variables affected my
    utility function, I skipped this step.

    Next step was to assign probability to each unknown state. Some of these
    were simple, like whether or not my friends are going out, if I have to
    work this week and the probability of going out on a Thursday. None of
    these variables depended on other variables, making figuring out the
    probability as easy as just remembering how often these things happen.

    With all the edge variables and decisions added, the more complex variables
    had to be created. These are also based mostly on empirical evidence. The
    most complex probability table was the \emph{Will finish exercise in time},
    which depended on when the deadline was, how far I had already gotten and
    whether or not I would go out on Friday. Since both deadline and progress
    had three values to choose from, making it an 18x2 big probability table. A
    small piece of it can be seen in Table \ref{tab:prob}.

    \begin{table}[ht!]
        \centering
        \begin{tabular}{| c | c | c | c |}\hline
            Progress & \multicolumn{3}{|c|}{Started}\\ \hline

            Go out on Friday & \multicolumn{3}{|c|}{Yes} \\ \hline

            Deadline & Two weeks & Next week & Tomorrow\\ \hline

            Yes & 0.95 & 0.8 & 0.05\\ \hline

            No & 0.05 & 0.2 & 0.95\\ \hline
        \end{tabular}
        \caption{Probability of finishing the exercise before deadline}
        \label{tab:prob}
    \end{table}




\section*{Results}
    \blindtext

\section*{Discussion}
    \blindtext

\bibliography{references}
\bibliographystyle{apacite}

\end{document}
