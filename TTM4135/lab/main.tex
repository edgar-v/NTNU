\documentclass[paper=a4, fontsize=11pt]{article}

\usepackage[utf8]{inputenc}
\usepackage{enumerate}
\usepackage{amsmath}
\usepackage{graphicx}
\usepackage{natbib}
\usepackage{blindtext}

\usepackage[margin=1cm]{caption}

\setlength{\parindent}{0.0in}
\setlength{\parskip}{0.1in}



\begin{document}
\mbox{}\\[3pc]
\begin{center}
    \Huge{Web security lab}\\[2pc]

    \Large{Edgar Vedvik}\\[1pc]
    \large{Group 42}\\[2pc]
    \large{\today}
\end{center}
\vspace{5cm}
\begin{center}
    Since this is an individual report, it follows that the individual lab
    exercises and this report were completed by myself alone.
\end{center}
\newpage


\section{Introduction}
    In this lab we were tasked with exploring many different aspects of
    information security. There were four main categories that we were to
    explore and solve challenges in. The first category was PGP, and the tasks
    included creating certificates, uploading them and signing and verifying
    messages. The second category was certificates and Certificate
    Authorities(CA's). The tasks here were to create personal and group
    certificates and using them to access websites. Our third task was to
    configure access control for an Apache webserver, setting up SSL and client
    authentication. Lastly we were tasked with securing a PHP application.

    All individual tasks were completed for the lab, this includes all the PGP
    tasks, and all CA tasks up until the group CA request.

    

\section{Questions}
    \textbf{Q1:} What are examples of a Certificate Hierarchy graph and Web
    of Trust graph in use in the world today? Are there specific organizations
    that prefer one or the other?\\

    \textbf{A1:} The strong set is the largest collection of
    strongly connected PGP keys, which means any two keys have a path between
    them. 

    \textbf{Q2:} How does the .p12 file work? Explain what this file is and its
    relationship to the NTNU CA\\

    \textbf{A2:} .p2 or PKCS \#12. is an archive format that can story multiple
    cryptography objects in the same file. Such as private key with its X.509
    certificate. The NTNU CA is the root CA for this.

    \textbf{Q3:} Comment on security related issues regarding the cryptographic
    algorithms used to generate and sign your group’s web server certificate
    (key lengths, algorithms, etc.).\\

    \textbf{A3:} The key length chosen was 4096 bits RSA. The key is within the
    recommended length.

    \textbf{Q4:} List and explain each of the verifications. What is the
    difference between the two files?  Who has signed the Apache release? What
    is the point of this process?\\

    \textbf{A4:} 
    \begin{enumerate}
        \item PGP signature. To verify this we need the public key from a
            key server. Once we have this we can verify the download with the
            key. If it matches, we know the file has not been tampered with.
            However, in order to trust the key we either need to meet the
            person face to face, or enter a web of trust.
        \item MD5 hash can be used to verify that the file is the same as the
            the one you intended to download. It does not say anything about
            who gave you this file or that the file is from apache.
    \end{enumerate}

    \textbf{Q5:} What is the purpose of a certificate chain? Describe, on a
    high-level, the steps that your browser takes when verifying the
    authenticity of a web page served over HTTPS.\\

    \textbf{A5:} Certificate chains are used to verify the authenticity of a
    certificate. The signature for each certificate is created with the private
    key of the next certificate in the chain, and a browser can then verify the
    certificate against the public key of the next certificate until it reaches
    a root CA.

    \textbf{Q6:} Web servers offering weak cryptography are subject to several
    attacks. List 3 specific feasible attacks that are possible due to weak
    cryptography and explain them briefly. How did you configure your server to
    prevent such attacks?\\

    \textbf{A6:} 
    \begin{enumerate}
        \item Factorisation of RSA modulus: RSA relies on the difficulty of
            factorizing large primes. In 2009, researches were able to
            factorize 768-bit RSA \cite{kleinjung}.
        \item Brute force attack: With more and more computational power, GPUs
            and FPGAs specifically designed to crack hashes, hashes with a low
            amount of keys, can be cracked very quickly.
        \item Chosen-Ciphertext attack (CCA). An attacker chooses a ciphertext
            to send to the server and receives a decrypted plain-text. By doing
            this enough times, the server leaks information that can be used to
            obtain the private key.
    \end{enumerate}

    \textbf{Q7:} Cookies can be a potential security risk if not handled
    properly, especially if they contain sensitive information. Two important
    flags can be set on cookies: HTTP-Only and secure. What does the HTTP-Only
    flag do and how does it work? What does the secure flag do and how does it
    work?\\

    \textbf{A7:} A HTTP-Only cookie cannot be accessed by client side APIs,
    such as Javascript and is therefore safe from XSS. It is however, still
    vulnerable to CSRF attacks. The secure cookie can only be transmitted over
    HTTPS, and can therefore not be eavesdropped upon.

    \textbf{Q8:} What kind of malicious attacks is your web application (PHP)
    vulnerable to? Describe four briefly, and point out what countermeasures
    you have developed in your code to prevent such attacks.\\

    \textbf{A8:} This task was not completed, but I have described them
    anyways:
    \begin{enumerate}
        \item SQL injections: Wherever user input is used to query a database
            we are vulnerable to SQL injections if we do not handle the input
            correctly.Do not escape and concatenate / interpolate data
            that goes into an SQL query yourself. Use prepared statements.
        \item Cross Site Request Forgery (CSRF): Execute unauthorized commands
            from a user that the website trusts. For example placing a
            malicious link in an <img> tag will cause the browser to open this
            site and potentially execute some action that the user did not
            authorize. To prevent this, use CSRF tokens or re-authentication.
        \item Cross site scripting (XSS): Inject malicious code in user
            supplied data on a webpage. If this is not escaped properly, the
            code will then be run whenever a user views the user supplied data.
            The solution is to sanitize the user input or to not allow certain
            characters or patterns.
        \item Code/Shell injection: When running user input through shell
            commands or eval() function a malicious attacker can supply code
            executes additional commands that you did not expect. Using eval is
            seen as bad practice and should be avoided. If shell commands are
            necessary, make sure the input to these functions are not tainted.

    \end{enumerate}

    \textbf{Q9:} Passwords should be salted and hashed. Why are both of these
    steps necessary? Explain how you have implemented these for storing
    passwords in your database.\\

    \textbf{A9:} Hash algorithms are one-way functions, therefore a hashed
    password cannot be reversed to the original password. This adds an
    additional security feature should the password database be exposed.
    Lookup tables and rainbow tables can still be used to crack passwords
    relatively quick since every hash of a password is the same. To mitigate
    this we introduce salting, which is the process of appending a random
    string of bytes to the password before hasing so that every hash of the
    same password will be different.

    I have not implemented this, but a good implementation would be to select a
    slow hash algorithm, such as bcrypt. The salt should be generated by a
    crypographically secure pseudo-random number generator. Add the salt to the
    password and then hash it. Save the hashed password and the salt in the
    database

    \bibliographystyle{chicago}
    \bibliography{references}

\end{document}
